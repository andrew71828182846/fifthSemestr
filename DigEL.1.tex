\documentclass[12pt]{article}
\usepackage[utf8x]{inputenc} 
\usepackage[russian]{babel} 
\usepackage[top=1in, bottom=1in, left=1in, right=1in]{geometry}
\usepackage[onehalfspacing]{setspace}
\usepackage{amsmath, amssymb, amsthm}
\usepackage{enumerate, enumitem}
\usepackage{fancyhdr, graphicx, proof, comment, multicol}
\usepackage[siunitx, RPvoltages]{circuitikz}
\usepackage[none]{hyphenat}
\usepackage{pgf,tikz,pgfplots}
\pgfplotsset{compat=1.15}
\usepackage{wrapfig}
\usepackage{mathrsfs}
\usepackage{pst-plot}
\usetikzlibrary{arrows}
\pagestyle{empty}% This command prevents hyphenation of words
\binoppenalty=\maxdimen % This command and the next prevent in-line equation breaks
\relpenalty=\maxdimen
\usepackage{microtype} % Modifies spacing between letters and words
\usepackage{mathpazo} % Modifies font. Optional package.
\usepackage{mdframed} % Required for boxed problems.
\usepackage{parskip} % Left justifies new paragraphs.
\usepackage{tikz}
\linespread{1.1} 
\pagestyle{fancy}
\lhead{\textbf{Цифровая электроника}}
\chead{\includegraphics[scale=0.18]{images.png}}
\rhead{\textbf{СПбГЭТУ «ЛЭТИ»}}

\newenvironment{problem}[1]
{\begin{mdframed}[linewidth=0.6pt]
        \textsc{Problem #1:}

}
    {\end{mdframed}}

\newenvironment{solution}
    {\textsc{Solution:}\\}
    {\newpage}% puts a new page after the solution
    
\newenvironment{statement}[1]
{\begin{mdframed}[linewidth=0.6pt]
        \textsc{Statement #1:}

}
    {\end{mdframed}}

%\newenvironment{prf}
 %   {\textsc{Proof:}\\}
 %   {\newpage}% puts a new page after the solution

\newcommand{\R}{\mathbb{R}}
\newcommand{\C}{\mathbb{C}}
\newcommand{\Z}{\mathbb{Z}}
\newcommand{\N}{\mathbb{N}}
\newcommand{\Q}{\mathbb{Q}}

\begin{document}
\vspace{25ex}
\noindent
\textbf{группа 8871} \hfill \textbf{Веренёв А.А.} \\
\normalsize Прокшин А.Н \hfill  \today \\

% This is where you name your homework
\begin{center}
\textbf{Практическая работа №1 (III-вар.)}
\end{center}
1) Первое задание:
\begin{center}
    \begin{tikzpicture}[scale = 2]
    \draw [thick] (-4,0)node [left] {${m}_1$}  -- (-2,3)node [above] {${m}_2$} -- (0,0)node [right] {${m}_3$} -- (-4,0);
    \draw [dashed] (-4.2,0.6)-- (0.2,0.6);
    \draw[very thick] (-1.7,0.6) -- (-1.7, 0)node [below] {$S_1$};
    \draw[very thick] (-1.7,0.6) -- (-3.1,1.35)node [above left ] {${S}_2$};
    \draw [thin] (-3.8,1.35) -- (-0.2, 1.35) ;
    \end{tikzpicture}    
\end{center}
2) Второе задание:
$$U=200B$$
$$alpha = 30^{\circ} $$

\begin{center}
\begin{tikzpicture}[scale =1.5 ]

   \newdimen\R
   \R=3cm
   \draw (0:\R) \foreach \x in {60,120,...,360} {  -- (\x:\R) };
   \foreach \x/\l/\p in
     { 60/{011}/above,
      120/{010}/above,
      180/{110}/left,
      240/{100}/below,
      300/{101}/below,
      360/{001}/right
     }
     \node[inner sep=1pt,circle,draw,fill,label={\p:\l}] at (\x:\R) {};
     \draw  (0,0) circle (2.6);
     \draw [-latex] (0,0) -- (30:2.6cm);
     \draw [thick,-latex] (0,0) -- (60:3cm);
     \draw [line width=1mm,-latex,green] (0,0) -- (60:1.5cm);
     \draw [thick,-latex] (0,0) -- (120:3cm);
     \draw [thick,-latex] (0,0) -- (180:3cm);
     \draw [thick,-latex] (0,0) -- (240:3cm);
     \draw [thick,-latex] (0,0) -- (300:3cm);
     \draw [thick,-latex] (0,0) -- (360:3cm);
     \draw [line width=1mm,-latex,red] (0,0) -- (360:1.5cm);
     \node [line width=2mm,blue] at (0,0) {\textbullet};
     \draw (1,0) arc (0:30:1);
     \node[] at (15:1.3)  {$30^{\circ}$};
     \draw [dashed] (30:2.6cm) -- (1.5,0);
     \draw [dashed] (30:2.6cm) -- (0.75,1.3);


\end{tikzpicture}
\end{center}

3) Третье задание: 
\begin{center}
\begin{tikzpicture}[scale=1.5]
\begin{axis}[
  samples at={-1,-0.5,...,3}, % gives a data point every pi/4
  xlabel = $t$,
  ylabel = $m$,
  axis lines = center, 
  enlargelimits,
  every axis x label/.append style = {below},
  every axis y label/.append style = {left},
  xtick = {-1, 1, 3},
  %xticklabels = {$-1$, $1$, $3$}, % not needed, numbers are set in math mode by default
  declare function = {
    T(\x) = (2 / pi) * rad(asin(sin(deg(pi * \x))));
  }
]
\addplot[ultra thick, black] {T(x)};
\draw[thick,draw=green,
      domain=-6.5:6.5,samples=300,variable=\x] 
      plot (\x,-1);
\draw[thick,draw=blue,
      domain=-6.5:6.5,samples=300,variable=\x] 
      plot (\x,0);
\draw[thick,draw=red,
      domain=-6.5:6.5,samples=300,variable=\x] 
      plot (\x,1);

\end{axis}
\end{tikzpicture}
\end{center}




\end{document}
