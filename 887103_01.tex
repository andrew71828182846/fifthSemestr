\providecommand{\pdfxopts}{a-1b,cyrxmp}
\providecommand{\thisyear}{2020}
\immediate\write18{rm \jobname.xmpdata}%  uncomment for Unix-based systems
\begin{filecontents*}{\jobname.xmpdata}
\Title{ЦЭ 887103.01 Э3 Веренёв Андрей Александрович \textemdash\thisyear}
\Author{Веренёв Андрей}
\Creator{pdfTeX + pdfx.sty with options \pdfxopts }
\Subject{Практическая работа №1 Вектороная ШИМ}
\Keywords{ШИМ, ЕСКД, PWM}
\CoverDisplayDate{май \thisyear}
\CoverDate{\today}
\Copyrighted{True}
\Copyright{Public Domain}
\CopyrightURL{http://github.com/trot-t}
\Creator{pdfTeX + pdfx.sty with options \pdfxopts }
\end{filecontents*}

\documentclass[russian,utf8,nocolumnxxxi,nocolumnxxxii]{eskdtext}

\pdfcompresslevel=9

\usepackage[\pdfxopts]{pdfx}[2016/03/09]
\PassOptionsToPackage{obeyspaces}{url}
\let\tldocrussian=1  % for live4ht.cfg
% http://texdoc.net/texmf-dist/doc/latex/pdfx/pdfx.pdf

%
% https://tavda.net/blog/latex/
% 

\usepackage[TS1,T2A]{fontenc}
\usepackage[utf8]{inputenc}
\usepackage[english,russian]{babel}
                                           % для чертежей
  % для электронных схем

\usepackage{enumerate, enumitem}
\usepackage{fancyhdr, graphicx, proof, comment, multicol}
\usepackage{pgf,tikz,pgfplots}
\usepackage{wrapfig}
\usepackage{mathrsfs}
\usepackage{pst-plot}
\usepackage{microtype} % Modifies spacing between letters and words
\usepackage{mathpazo} % Modifies font. Optional package.
\usepackage{mdframed} % Required for boxed problems.
\usepackage{parskip} % Left justifies new paragraphs.
\usepackage{tikz}
\usepackage{amsmath,amsfonts}
\usepackage{amssymb}
%\usepackage[scr]{rsfso} % буквы для алгебры множеств

\usepackage{gnuplottex}   % автоматическая вставка графиков из программ моделирования электрических схем

\usepackage{enumitem}

%%% Межстрочный интервал
\usepackage{setspace}

%% таблицы в стиле старых книг
\usepackage{booktabs} 

%% для подкладывания отдельных pdf-страниц 
\usepackage{pdfpages}

%% для кода
\usepackage{listingsutf8}

\definecolor{lightgrey}{rgb}{0.9,0.9,0.9}
\definecolor{lightblue}{rgb}{0,0,1}

\definecolor{grey}{rgb}{0.5,0.5,0.5}
\definecolor{blue}{rgb}{0,0,1}
\definecolor{violet}{rgb}{0.5,0,0.5}

\definecolor{darkred}{rgb}{0.5,0,0}
\definecolor{darkblue}{rgb}{0,0,0.5}
\definecolor{darkgreen}{rgb}{0,0.5,0}


\lstset{%
  language=C++,%
  morekeywords={constexpr,nullptr,size_t,uint32_t,assert,override,final},%
  basicstyle=\ttfamily\footnotesize,%
  sensitive=true,%
  keywordstyle=\color{blue},%
  stringstyle=\color{darkgreen},%
  commentstyle=\color{violet},%
  showstringspaces=false,%
  tabsize=2,%
  frame=leftline,
  rulecolor=\color{lightblue},
  xleftmargin=10pt,
}

\lstset{
extendedchars=\true,
%inputencoding=utf8x,   
  numberstyle=\tiny,
  numbers=left,
  numbersep=10pt,
  xleftmargin=10pt,
  framesep=4.5mm,
  framexleftmargin=2.5mm,
  framexleftmargin=5pt,
  framesep=15pt,
  fillcolor=\color{lightgrey},
}

\def\No{\textnumero} % №


%\ESKDcompany{Санкт-Петербургский государственный электротехнический университет "ЛЭТИ"}
\ESKDtitle{отчет по практической работе}
\ESKDsignature{ШИМ}
\ESKDauthor{\scalebox{0.9}[1.0]{Веренёв А.А.}}
\ESKDchecker{\scalebox{0.9}[1.0]{Прокшин~А.Н.}}
\ESKDdocName{Веренев Андрей Александрович}

\ESKDdocName{Векторная ШИМ}
\ESKDsignature{ПРАКТИЧЕСКАЯ РАБОТА}
 
\ESKDdepartment{%
Государственное образовательное учреждение высшего профессионального образования
%кафедра РАПС
}
\ESKDcompany{%
<<Санкт-Петербургский государственный электротехнический университет "ЛЭТИ">> 
}
\ESKDauthor{\scalebox{0.9}[1.0]{Веренёв А.А.}} % "Разраб." в штампе на листе содержания
\ESKDchecker{\scalebox{0.9}[1.0]{Прокшин А.Н.}} % "Пров."  в штампе на листе содержания
\ESKDnormContr{\scalebox{0.9}[1.0]{Веренёв А.А..}} % "Н. контр." в штампе на листе содержания
\ESKDapprovedBy{\scalebox{0.9}[1.0]{Прокшин А.Н.}}%  "Увт." в штампе на листе содержания
\ESKDdate{2020/05/15} % Дата (Год отображается на титульной странице) 
\ESKDsignature{Цифровая Электроника 887103.01 Э3} % Шифр
\ESKDletter{}{У}{} % Литеры, для учебных работ используется литера У
 
\renewcommand{\ESKDtheTitleFieldX}{%
Санкт-Петербург
 
\ESKDtheYear~г.} % Шаблон для отображения в нижней части титульного листа города и года 
%\author{}
% Конец преамбулы

\begin{document}
\maketitle

% шаблон графика параболы
%\begin{tikzpicture}
%\newcommand{\xb}{-3}
%\newcommand{\xa}{3}
%\draw[thin, ->] (-6,0) -- (6,0) node[right] {$X$};
%\draw[thin, ->] (0,-6) -- (0,6) node[left] {$Y$};
%\foreach \x\xtext in {-5/-5,5/5,{\xb}/\xb,{\xa}/{\displaystyle \frac{-b+\sqrt{b^2-4ac}}{2a}}} % 
%   \draw (\x,0.1) -- (\x,-0.1) node[below] {$\xtext$};
%\draw[domain=-5:5, help lines, smooth]
%       plot ({\x},{0.2*(\x-\xa)*(\x-\xb)});
%\end{tikzpicture}

% грязный хак

1) Первое задание:
\begin{center}
    \begin{tikzpicture}[scale = 2]
    \draw [line width=0.5mm] (0,0)node [left] {${m}_1$}  -- (2,3.46)node [above] {${m}_2$} -- (4,0)node [right] {${m}_3$} -- (0,0);
    \draw [line width=0.3mm, -latex] (0,0) -- (xyz polar cs:angle=30,radius=2.5) node [above,minimum size=1.2cm] {$\overline{U}$};
    \draw[dashed] (xyz polar cs:angle=30,radius=2.5) -- +(xyz polar cs:angle=120,radius=2);
    \draw[dashed] (xyz polar cs:angle=30,radius=2.5) -- +(xyz polar cs:angle=-60,radius=2);
    \draw[dashed] (xyz polar cs:angle=30,radius=2.5) -- +(180:2);
    \draw[dashed] (xyz polar cs:angle=30,radius=2.5) -- +(0:2);
    \draw[dashed] (xyz polar cs:angle=30,radius=2.5) -- +(-120:2);
    \draw[dashed] (xyz polar cs:angle=30,radius=2.5) -- +(60:2);
    \draw (xyz polar cs:angle=30,radius=2.5) -- +(-90:1.25)node [below,minimum size=1.2cm] {${U}_A$};
    \draw (xyz polar cs:angle=30,radius=2.5) -- +(150:1.25)node [above left] {${U}_C$};
    \node[] at (2.6, 3)  {$m_1$};
    \node[] at (3.3, 1.8)  {$m_2$};
    \node[] at (4, 0.7)  {$m_3$};
    %\draw [dashed] (-4.2,0.6)-- (0.2,0.6);
    %\draw[very thick] (-1.7,0.6) -- (-1.7, 0)node [below] {$S_1$};
    %\draw[very thick] (-1.7,0.6) -- (-3.1,1.35)node [above left ] {${S}_2$};
    %\draw [thin] (-3.8,1.35) -- (-0.2, 1.35) ;
    \end{tikzpicture}    
\end{center}
2) Второе задание:
$$U=200B; \;alpha = 30^{\circ}, \; f= 5 kHz$$

\begin{center}
\begin{tikzpicture}[scale =1.5 ]

   \newdimen\R
   \R=3cm
   \draw (0:\R) \foreach \x in {60,120,...,360} {  -- (\x:\R) };
   \foreach \x/\l/\p in
     { 60/{011}/above,
      120/{010}/above,
      180/{110}/left,
      240/{100}/below,
      300/{101}/below,
      360/{001}/right
     }
     \node[inner sep=1pt,circle,draw,fill,label={\p:\l}] at (\x:\R) {};
     \draw  (0,0) circle (2.6);
     \draw [-latex] (0,0) -- (30:2.1cm);
     \draw [thick,-latex] (0,0) -- (60:2.6cm);
     \draw [line width=1mm,-latex,green] (0,0) -- (60:1.3cm);
     \draw [thick,-latex] (0,0) -- (120:3cm);
     \draw [thick,-latex] (0,0) -- (180:3cm);
     \draw [thick,-latex] (0,0) -- (240:3cm);
     \draw [thick,-latex] (0,0) -- (300:3cm);
     \draw [thick,-latex] (0,0) -- (360:3cm);
     \draw [line width=1mm,-latex,red] (0,0) -- (360:1.3cm);
     \node [line width=2mm,blue] at (0,0) {\textbullet};
     \draw (1,0) arc (0:30:1);
     \node[] at (15:1.3)  {$30^{\circ}$};
     \draw [dashed] (30:2.1cm) -- (1.3,0);
     \draw [dashed] (30:2.1cm) -- (0.6,1.05);


\end{tikzpicture}
\end{center}

3) Третье задание: 
\begin{center}
\begin{tikzpicture}[scale = 1.5]
\draw[very thick] (0,0) -- (4,4) -- (4,0) -- (0,0);
\draw[dashed] (0,3.6) -- (4.4,3.6)node [above left] {$A$};
\draw[dashed] (0,2) -- (4.4,2)node [above left] {$B$};
\draw[dashed] (0,0.4) -- (4.4,0.4)node [above left] {$C$};
\draw [dashed] (0.4,0.4) -- (0.4,-0.5);
\draw[dashed] (2,2) -- (2,-1);
\draw[dashed] (3.6,3.6) -- (3.6,-1.5);
\draw[dashed] (4,0) -- (4,-1.5);
\draw [line width=0.5mm, green, <->] (0.4,-0.5) -- (4,-0.5)node [pos=0.5, above] {$T_A$};
\draw [line width=0.5mm, blue, <->] (2,-1) -- (4,-1)node [pos=0.5, above] {$T_B$};
\draw [line width=0.3mm, red, <->] (3.6,-1.5) -- (4,-1.5)node [pos=0.5, above] {$T_C$};

\end{tikzpicture}
\end{center}



\begin{center}
\begin{tikzpicture}[scale=1.2]
\begin{axis}[
  samples at={-1,-0.5,...,3}, % gives a data point every pi/4
  xlabel = $t$,
  ylabel = $m$,
  axis lines = center, 
  enlargelimits,
  every axis x label/.append style = {below},
  every axis y label/.append style = {left},
  xtick = {-1, 1, 3},
  %xticklabels = {$-1$, $1$, $3$}, % not needed, numbers are set in math mode by default
  declare function = {
    T(\x) = (2 / pi) * rad(asin(sin(deg(pi * \x))));
  }
]
\addplot[ultra thick, black] {T(x)};

      
\addplot [very thick,
    domain=-3:3, 
    samples=100, 
    color=blue,
]
{y};
\addplot [very thick,
    domain=-3:3, 
    samples=100, 
    color=red,
]
{y+1};
\addplot [very thick,
    domain=-3:3, 
    samples=100, 
    color=green,
]
{y-1};
      

\end{axis}
\end{tikzpicture}
\end{center}
 
\end{document}